% Created 2020-11-19 Thu 17:46
% Intended LaTeX compiler: pdflatex
\documentclass[presentation]{beamer}
\usepackage[utf8]{inputenc}
\usepackage[T1]{fontenc}
\usepackage{graphicx}
\usepackage{grffile}
\usepackage{longtable}
\usepackage{wrapfig}
\usepackage{rotating}
\usepackage[normalem]{ulem}
\usepackage{amsmath}
\usepackage{textcomp}
\usepackage{amssymb}
\usepackage{capt-of}
\usepackage{hyperref}
\usetheme{default}
\author{Kavin Desi Valli}
\date{\today}
\title{Dart Notes}
\hypersetup{
 pdfauthor={Kavin Desi Valli},
 pdftitle={Dart Notes},
 pdfkeywords={},
 pdfsubject={},
 pdfcreator={Emacs 27.1 (Org mode 9.4)}, 
 pdflang={English}}
\begin{document}

\maketitle
\begin{frame}{Outline}
\tableofcontents
\end{frame}


\begin{frame}[label={sec:orgd901e37}]{Introduction}
It is a lot like Javascript just that a syntax is a little different.
\end{frame}

\begin{frame}[label={sec:orgaf6b74a}]{DartPad}
An online pad can be found at \href{https://dartpad.dev/}{DartPad}
\end{frame}

\begin{frame}[label={sec:orge856a76},fragile]{Main}
 \begin{verbatim}

void main() {
  print("Hello World!");
}

\end{verbatim}
\end{frame}

\begin{frame}[label={sec:orgc87552c},fragile]{Variables}
 \begin{block}{Types}
\begin{itemize}
\item Integer : \texttt{int}
\item String : \texttt{String}
\item Boolean : \texttt{bool}
\item Dynamic : \texttt{dynamic}
\item List: \texttt{List}
\item Void: \texttt{void} (returns nothing)
\end{itemize}
\end{block}

\begin{block}{Usage}
\begin{verbatim}

dynamic main() {
  int age = 30;
  print(age);
  String name = 'chul-li';
  print(name);
  bool isNight = false;
  print(isNight);
  dynamic dyvar = 'chun-li';
  dyvar = 30;
  print(dyvar);
}

\end{verbatim}

\begin{block}{Output}
\begin{verbatim}

30
chul-li
false
30

\end{verbatim}
\end{block}
\end{block}

\begin{block}{Lists}
\begin{block}{Basic}
\begin{verbatim}

void main() {
  List names = ['chun-li', 'yoshi', 'mario'];
  print(names);
}

\end{verbatim}
\end{block}

\begin{block}{Adding to a list}
\begin{verbatim}

names.add('luigi');

\end{verbatim}
\end{block}

\begin{block}{Remove from list}
\begin{verbatim}

names.remove('yoshi');

\end{verbatim}
\end{block}

\begin{block}{Specific Type List}
\begin{verbatim}

List<String> names = ['hello','world']

\end{verbatim}
\end{block}
\end{block}
\begin{block}{Classes}
\begin{verbatim}

void main() {
  // Instantiating a class
  User userOne = User('luigi', 30);
  print(userOne.username);
  userOne.login();

  User userTwo = User('Kavin', 14);
  print(userTwo.username);

  SuperUser userThree = SuperUser('Yoshi', 20);
  print(userThree.username);
  userThree.publish();
  userThree.login();

  // BUG: This won't work
  userTwo.publish();
}

class User {
  String username;
  int age;

  User(String username, int age) {
    this.username = username;
    this.age = age;
  };

  void login() {
    print("User logged in");
  }

}

\end{verbatim}
\begin{block}{Extend Classes}
\begin{verbatim}

class SuperUser extends User {

  //INFO: Constructor: Don't set the values. We inherit from User object. So use =super=
  SuperUser(String username, int age) : super(username, age)

  void publish() {
    print("published update");
  }
}

\end{verbatim}
\end{block}
\end{block}
\begin{block}{Important!!}
\begin{block}{We cannot change type of variable}
\begin{verbatim}

void main() {
  String name = 'chul-li';
  name = 30;
  print(name);
}

\end{verbatim}
\texttt{A value of type 'int' can't be assigned to a variable of type 'String'}
\end{block}

\begin{block}{We can change the variable's value}
\begin{verbatim}

void main() {
  String name = 'chul-li';
  name = 'kavin';
  print(name);
}

\end{verbatim}
\texttt{kavin}
\end{block}
\end{block}
\end{frame}

\begin{frame}[label={sec:orgd957af2},fragile]{Functions}
 \begin{block}{Main Function}
Top level required function in dart
\begin{verbatim}

void main() {

}

\end{verbatim}
\texttt{void} means function is not gonna return anything.
\end{block}

\begin{block}{Return what in function?}
\begin{verbatim}

void main() {
  print("something");
  String greet = greeting();
  print(greet);
  int age = getAge();
  print(age);
}

String greeting() {
  return 'hello';
}

int getAge() {
  return 30;
}

\end{verbatim}
\end{block}

\begin{block}{Arrow functions}
\begin{verbatim}

String greeting() => 'hello';
int getAge() => 30;

\end{verbatim}
\end{block}
\end{frame}
\end{document}
